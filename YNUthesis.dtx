% \iffalse meta-comment
% !TeX program  = XeLaTeX
% !TeX encoding = UTF-8
%
% Copyright (C) 2023 by Rui-Zhi Li <liruizhi0871@gmail.com>
%--------------------------------------------------------------------------
%
% This work may be distributed and/or modified under the
% conditions of the LaTeX Project Public License, either
% version 1.3c of this license or (at your option) any later
% version. This version of this license is in
%    http://www.latex-project.org/lppl/lppl-1-3c.txt
% and the latest version of this license is in
%    http://www.latex-project.org/lppl.txt
% and version 1.3 or later is part of all distributions of
% LaTeX version 2005/12/01 or later.
%
% This work has the LPPL maintenance status "maintained".
%
% --------------------------------------------------------------------------
%
%<*internal>
\iffalse
%</internal>
%
%<*readme>
# YNUthesis

---
Copyright (C) 2023 by Rui-Zhi Li <liruizhi0871@gmail.com>.
%</readme>
%
%<*internal>
\fi
\begingroup
  \def\NameOfLaTeXe{LaTeX2e}
\expandafter\endgroup\ifx\NameOfLaTeXe\fmtname\else
\csname fi\endcsname
%</internal>
%
%<*install>
\input docstrip.tex
\keepsilent
\askforoverwritefalse
\usedir{tex/latex/YNUthesis}
\preamble

Copyright (C) 2023 by Rui-Zhi Li <liruizhi0871@gmail.com>
---------------------------------------------------------------------

This work may be distributed and/or modified under the
conditions of the LaTeX Project Public License, either
version 1.3c of this license or (at your option) any later
version. This version of this license is in
   http://www.latex-project.org/lppl/lppl-1-3c.txt
and the latest version of this license is in
   http://www.latex-project.org/lppl.txt
and version 1.3 or later is part of all distributions of
LaTeX version 2005/12/01 or later.

This work has the LPPL maintenance status "maintained".

The Current Maintainer of this work is Rui-Zhi Li.

This work consists of the files YNUthesis.dtx,
          and the derived files YNUthesis.ins,
                                YNUthesis.cls,
                            and README.md.
---------------------------------------------------------------------

\endpreamble

\generate{
  \usedir{tex/latex/YNUthesis}
    \file{\jobname.cls}        {\from{\jobname.dtx}{class}}
%</install>
%<*internal>
  \usedir{source/latex/YNUthesis}
    \file{\jobname.ins}        {\from{\jobname.dtx}{install}}
%</internal>
%<*install>
  \usedir{doc/latex/YNUthesis}
  \nopreamble\nopostamble
    \file{README.md}           {\from{\jobname.dtx}{readme}}
}

\obeyspaces
\Msg{*************************************************************}
\Msg{*                                                           *}
\Msg{* To finish the installation you have to move the following *}
\Msg{* files into a directory searched by TeX:                   *}
\Msg{*                                                           *}
\Msg{* The recommended directory is TDS:tex/latex/fduthesis      *}
\Msg{*                                                           *}
\Msg{*     YNUthesis.cls                                         *}
\Msg{*                                                           *}
\Msg{* To produce the documentation, run the file fduthesis.dtx  *}
\Msg{* through XeLaTeX.                                          *}
\Msg{*                                                           *}
\Msg{* Happy TeXing!                                             *}
\Msg{*                                                           *}
\Msg{*************************************************************}

\endbatchfile
%</install>
%
%<*internal>
\fi
%</internal>
%
%<*driver>
%\ProvidesFile{YNUthesis.dtx}
%</driver>
%<class>\NeedsTeXFormat{LaTeX2e}[2020/01/01]
%<class>\ProvidesExplClass{\ExplFileName}
%<*class>
%   [2023/01/01 v0.0.1 Thesis-template-for-Yunnan-University]
%</class>
%
%<*driver>
\documentclass{ctxdoc}
\EnableCrossrefs
\CodelineIndex
\RecordChanges
\begin{document}
  \DocInput{\jobname.dtx}
\end{document}
%</driver>
% \fi
%
% \CheckSum{0}
%
% \CharacterTable
%  {Upper-case    \A\B\C\D\E\F\G\H\I\J\K\L\M\N\O\P\Q\R\S\T\U\V\W\X\Y\Z
%   Lower-case    \a\b\c\d\e\f\g\h\i\j\k\l\m\n\o\p\q\r\s\t\u\v\w\x\y\z
%   Digits        \0\1\2\3\4\5\6\7\8\9
%   Exclamation   \!     Double quote  \"     Hash (number) \#
%   Dollar        \$     Percent       \%     Ampersand     \&
%   Acute accent  \'     Left paren    \(     Right paren   \)
%   Asterisk      \*     Plus          \+     Comma         \,
%   Minus         \-     Point         \.     Solidus       \/
%   Colon         \:     Semicolon     \;     Less than     \<
%   Equals        \=     Greater than  \>     Question mark \?
%   Commercial at \@     Left bracket  \[     Backslash     \\
%   Right bracket \]     Circumflex    \^     Underscore    \_
%   Grave accent  \`     Left brace    \{     Vertical bar  \|
%   Right brace   \}     Tilde         \~}
%
% \changes{v1.0}{hYYYY i/hMM i/hDDi}{Initial version}
%
% \GetFileInfo{YNUthesis.dtx}
%
% \DoNotIndex{hlist of control sequencesi}
%
% \title{\bfseries \CTeX{} 宏集手册}
% \author{\href{http://www.ctex.org}{CTEX.ORG}}
% \date{\filedate\qquad\fileversion\thanks{\ctexkitrev{\ExplFileVersion}.}}
% \maketitle
%
% \begin{abstract}
% \CTeX{} 宏集是面向中文排版的通用 \LaTeX{} 排版框架,为中文 \LaTeX{} 文档
% 提供了汉字输出支持、标点压缩、字体字号命令、标题文字汉化、中文版式调整、数字
% 日期转换等支持功能,可适应论文、报告、书籍、幻灯片等不同类型的中文文档。
%
% \CTeX{} 宏集支持 \LaTeX、\pdfLaTeX、\XeLaTeX、\LuaLaTeX、\upLaTeX{} 等多种不同
% 的编译方式,并为它们提供了统一的界面。主要功能由宏包 \pkg{ctex} 以及中文文档类
% \cls{ctexart}、\cls{ctexrep}、\cls{ctexbook} 和 \cls{ctexbeamer} 实现。
% \end{abstract}
%
% \tableofcontents
%
% \bigskip
% \setlength{\parskip}{0.8ex}
%
% \begin{documentation}
%
% \section{介绍}
%
% \end{documentation}
%
% \StopEventually{}
%
%\begin{implementation}
% \clearpage
% \section{代码实现}
%
%    \begin{macrocode}
%<@@=ctex>
%    \end{macrocode}
%
% \changes{v2.5.9}{2022/05/27}{设置消息模块的名字和类型。}
% \changes{v2.5.10}{2022/06/10}{更新一些内部函数。}
% \changes{v2.5.10}{2022/07/08}{不直接依赖 \pkg{xparse} 和 \pkg{l3keys2e}。}
%
%    \begin{macrocode}
%<*class|style>
\cs_if_exist:NF \NewDocumentCommand
  { \RequirePackage { xparse } }
%<class>\prop_gput:Nnn \g_msg_module_type_prop { ctex } { Class }
%<article>\prop_gput:Nnn \g_msg_module_name_prop { ctex } { ctexart }
%<book>\prop_gput:Nnn \g_msg_module_name_prop { ctex } { ctexbook }
%<report>\prop_gput:Nnn \g_msg_module_name_prop { ctex } { ctexrep }
%<beamer>\prop_gput:Nnn \g_msg_module_name_prop { ctex } { ctexbeamer }
%<ctexsize>\prop_gput:Nnn \g_msg_module_name_prop { ctex } { ctexsize }
%<ctexheading>\prop_gput:Nnn \g_msg_module_name_prop { ctex } { ctexheading }
%</class|style>
%    \end{macrocode}
%
%    \begin{macrocode}
%<*class|ctex>
%    \end{macrocode}
%
% \changes{v2.3}{2015/12/20}{与 \LaTeXiii{} (2015/12/20) 同步。}
% \changes{v2.4.10}{2017/07/19}{常数 \cs{c_minus_one} 已过时。}
% \changes{v2.4.10}{2017/07/22}{使用 \texttt{lazy} 函数对 Boolean 表达式
% 进行最小化运算(\LaTeXiii{} 2017/07/19)。}
%
% 检查 \pkg{expl3} 的版本。
%    \begin{macrocode}
\msg_new:nnnn { ctex } { l3-too-old }
  { Support~package~`#1'~too~old. }
  {
    Please~update~an~up-to-date~version~of~the~bundles\\\\
    `l3kernel'~and~`l3packages'\\\\
    using~your~TeX~package~manager~or~from~CTAN.
  }
\@ifpackagelater { expl3 } { 2021/02/10 } { }
  { \msg_error:nnn { ctex } { l3-too-old } { expl3 } }
%    \end{macrocode}
%
% \begin{variable}{\c_@@_engine_str,\c_@@_engine_file_str}
% 引擎检查。目前 \LaTeXiii{} 将 \ApTeX{} 识别为 \upTeX。
%    \begin{macrocode}
\str_const:Nx \c_@@_engine_str
  { \cs_if_exist:NTF \ngostype { aptex } { \c_sys_engine_str } }
\msg_new:nnnn { ctex } { engine-not-supported }
  { Engine~`#1'~is~not~yet~supported,~ctex~will~abort! }
  { You~can~switch~to~xelatex,~lualatex,~pdflatex,~uplatex,~or~aplatex. }
\file_if_exist:nTF { ctex-engine- \c_@@_engine_str .def }
  {
    \str_const:Nx \c_@@_engine_file_str
      { ctex-engine- \c_@@_engine_str .def }
  }
  { \msg_critical:nnx { ctex } { engine-not-supported } { \c_@@_engine_str } }
%    \end{macrocode}
% \end{variable}
%
%    \begin{macrocode}
%</class|ctex>
%    \end{macrocode}
%
% \changes{v2.5}{2020/04/19}
%   {处理 \cs{ctex_file_input:n} 在 \pkg{ctexsize} 中未定义的错误。}
% \changes{v2.5}{2020/04/21}{在 \pkg{ctexsize} 也载入 \pkg{fix-cm}。}
%
%    \begin{macrocode}
%<*class|ctex|ctexheading|ctexsize>
%    \end{macrocode}
%
% \pkg{ctexsize} 也要载入 \pkg{fix-cm} 包解决传统 cm 字体字号缺失的问题。
%    \begin{macrocode}
%<!ctexsize>\RequirePackage { ctexhook , ctexpatch }
%<!ctexheading>\RequirePackage { fix-cm }
%    \end{macrocode}
%
% \end{implementation}
%
% \Finale
%
\endinput