% !TeX program  = XeLaTeX
% !TeX encoding = UTF-8
\documentclass{YNUthesis}
\usepackage{YNUlogo}
\usepackage{zhlipsum}

\YNUsetup{
    style = {
        font = times,
    }
}

\begin{document}

% 这个命令用来关闭版心底部强制对齐,可以减少不必要的 underfull \vbox 提示,但会影响排版效果
% \raggedbottom

% 前置部分包含目录、中英文摘要以及符号表等
\frontmatter

% 目录
\tableofcontents
% 插图目录
\listoffigures
% 表格目录
% \listoftables

\begin{abstract}
  中文摘要
\end{abstract}

\begin{abstract*}
  English abstract
\end{abstract*}

% 符号表
% 语法与 LaTeX 表格一致:列用 & 区分,行用 \\ 区分
% 如需修改格式,可以使用可选参数:
%   \begin{notation}[ll]
%     $x$ & 坐标 \\
%     $p$ & 动量
%   \end{notation}
% 可选参数与 LaTeX 标准表格的列格式说明语法一致
% 这里的 “ll” 表示两列均为自动宽度,并且左对齐
\begin{notation}[ll]
  $x$                  & 坐标        \\
  $p$                  & 动量        \\
\end{notation}

% 主体部分是论文的核心
\mainmatter

% 建议采用多文件编译的方式
% 比较好的做法是把每一章放进一个单独的 tex 文件里,并在这里用 \include 导入,例如
%   \include{chapter1}
%   \include{chapter2}
%   \include{chapter3}

\chapter{介绍}
\zhlipsum[10]
\[
    a^2+b^2=c^2
\]

\section{动力学演化}
\zhlipsum[10]

\subsection{动力学演化}
\zhlipsum[10]

\subsubsection{动力学演化}
\zhlipsum[10]

\chapter{总结与展望}

% 后置部分包含参考文献、声明页(自动生成)等
\backmatter

% 打印参考文献列表
\printbibliography

\end{document}